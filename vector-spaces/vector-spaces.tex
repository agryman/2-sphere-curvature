\documentclass[11pt, oneside]{article}

\usepackage{../shared/preamble}
\addbibresource{../shared/references.bib}

\usepackage{../topological-spaces/topological-spaces}
\usepackage{../real-numbers/real-numbers}
\usepackage{vector-spaces}

\title{Vector Spaces}
\author{Arthur Ryman, {\tt arthur.ryman@gmail.com}}
\date{\today}

\begin{document}

\maketitle

\begin{abstract}
This article contains Z Notation type declarations for vector spaces and some related objects.
It has been type checked by \fuzz.
\end{abstract}

\section{Introduction}

Real vector spaces are multidimensional generalizations of real numbers.
They are the objects studied in linear algebra and are foundational to differential geometry.

\section{Real $n$-tuples}

\subsection{\zcmd{Rinf}}

Let $n$ be a natural number.
A finite sequence of $n$ real numbers is called a real $n$-tuple.
Let $\Rinf$ denote the set of all real $n$-tuples for any $n$.

\begin{zed}
	\Rinf == \seq \R
\end{zed}

\subsection{\zcmd{Rtuples}}

Let $\Rtuples(n)$ denote $\R^n$, the set of all $n$-tuples for some given $n$.
\begin{axdef}
	\Rtuples: \nat \fun \power \Rinf
\where
	\forall n: \nat @ \\
	\t1	\Rtuples(n) = \{~ v: \Rinf | \# v = n ~\}
\end{axdef}

\begin{remark}

\begin{zed}
	\Rinf = \bigcup \{~ n: \nat @ \Rtuples(n) ~\}
\end{zed}

\end{remark}

\subsubsection{\zcmd{pi}}

The real numbers that comprise an $n$-tuple are called its components.
The real number $v(i)$ is the $i$-th component of the $n$-tuple $v$ where
$1 \le i \le n$.
Let $\pi(i)$ be the projection function that maps an $n$-tuple $v$ to its $i$-th component $v(i)$.

\begin{axdef}
	\pi: \nat_1 \fun \Rinf \pfun \R
\where
	\forall i: \nat_1 @ \\
	\t1	\pi(i) = (\lambda v: \Rinf | i \in \dom v @ v(i))
\end{axdef}

\section{Scalar Multiplication}

\subsection{\zcmd{smulR}}

Let $v$ be an $n$-tuple and let $c$ be a real number.
Scalar multiplication of $v$ by $c$ is the $n$-tuple $c \smulR v$ defined by component-wise multiplication.

\begin{axdef}
	\_ \smulR \_ : \R \cross \Rinf \fun \Rinf 
\where
	\forall c: \R @ \\
	\t1	c \smulR \langle \rangle = \langle \rangle
\also
	\forall c: \R; n: \nat_1 @ \\
	\t1	\forall v: \Rtuples(n); i: 1 \upto n @ \\
	\t2		(c \smulR v)(i) = c \mulR v(i)
\end{axdef}

\section{Vector Addition and Subtraction}

\subsection{\zcmd{vaddR}}

Let $v$ and $w$ be $n$-tuples.
Vector addition of $v$ and $w$ is the $n$-tuple $v \vaddR w$ defined by component-wise addition.

\begin{axdef}
	\_ \vaddR \_: \Rinf \cross \Rinf \pfun \Rinf
\where
	\dom(\_ \vaddR \_) = \{~ v, w: \Rinf | \# v = \# w ~\}
\also
	\langle \rangle \vaddR \langle \rangle = \langle \rangle
\also
	\forall n: \nat_1 @ \\
	\t1	\forall v, w: \Rtuples(n); i: 1 \upto n @ \\
	\t2		(v \vaddR w)(i) = v(i) \addR w(i)
\end{axdef}

\subsection{\zcmd{vsubR}}

Vector subtraction is defined similarly.

\begin{axdef}
	\_ \vsubR \_: \Rinf \cross \Rinf \pfun \Rinf
\where
	\dom(\_ \vsubR \_) = \{~ v, w: \Rinf | \# v = \# w ~\}
\also
	\langle \rangle \vsubR \langle \rangle = \langle \rangle
\also
	\forall n: \nat_1 @ \\
	\t1	\forall v, w: \Rtuples(n); i: 1 \upto n @ \\
	\t2		(v \vsubR w)(i) = v(i) \subR w(i)
\end{axdef}

Each $\Rtuples(n)$ is a real vector space under the operations of scalar multiplication and vector addition
defined above. 

\section{Linear Transformations}

\subsection{Linear}

Let $n$ and $m$ be natural numbers.
A mapping $L$ from $\R^n$ to $\R^m$ is said to be a linear transformation if it preserves scalar multiplication and vector addition.
\begin{schema}{Linear}
	n, m: \nat \\
	L: \Rinf \pfun \Rinf
\where
	L \in \Rtuples(n) \fun \Rtuples(m)
\also
	\forall c: \R; v: \Rtuples(n) @ \\
	\t1	L(c \smulR v) = c \smulR L(v)
\also
	\forall v, w: \Rtuples(n) @ \\
	\t1	L(v \vaddR w) = L(v) \vaddR L(w)
\end{schema}

\subsection{\zcmd{linR}}

Define $\linR(n,m)$ to be the set of all linear transformations from $\R^n$ to $\R^m$.
\begin{axdef}
	\linR: \nat \cross \nat \fun \power(\Rinf \pfun \Rinf)
\where
	\forall n,m: \nat @ \\
	\t1	\linR(n,m) = \{~ L: \Rinf \pfun \Rinf | Linear ~\}
\end{axdef}

\section{The Dot Product}

\subsection{\zcmd{dotR}}

The {\it inner} or {\it dot} product of $n$-tuples $v$ and $w$ is the real number $v \dotR w$ defined by the sum of the component-wise products.

\begin{axdef}
	\_ \dotR \_ : \Rinf \cross \Rinf \pfun \R
\where
	\dom(\_ \dotR \_) = \{~ v, w: \Rinf | \# v = \# w ~\}
\also
	\langle \rangle \dotR \langle \rangle = \zeroR
\also
	\forall x, y: \R; v, w: \Rinf | \# v = \# w @ \\
	\t1	(\langle x \rangle \cat v) \dotR (\langle y \rangle \cat w) = x \mulR y \addR v \dotR w
\end{axdef}

Each $\Rtuples(n)$ is a real inner product space under the operation of dot product defined above.

\section{The Norm}

\subsection{\zcmd{normR}}

The norm $\norm{v}$ of the $n$-tuple $v$ is the positive square root of its dot product with itself.
$$
	\norm{v} = \sqrt{v \dotR v}
$$

Define $\normR(v)$ to be $\norm{v}$.
\begin{axdef}
	\normR: \Rinf \fun \R
\where
	\forall v: \Rinf @ \\
	\t1	\normR(v) = \sqrtR(v \dotR v)
\end{axdef}

The concepts of continuity, limits, and differentiability extend to functions between normed vector spaces such as $\R^n$.

\subsection{\zcmd{ballRn}}

Let $\ballRn(n,v,r)$ denote the open ball in $\Rtuples(n)$ of radius $r  \in \R$ centred at $v \in \Rtuples(n)$.

\begin{axdef}
	\ballRn: \nat \cross \Rinf \cross \R \pfun \power \Rinf
\where
	\forall n: \nat; v: \Rinf; r: \R | v \in \Rtuples(n) @ \\
	\t1	\ballRn(n, v, r) = \{~ w: \Rtuples(n) | \normR(v \vsubR w) \ltR r ~\}
\end{axdef}

\subsection{\zcmd{ballsRn}}

Let $\ballsRn(n)$ denote the family of all open balls in $\Rtuples(n)$.

\begin{axdef}
	\ballsRn: \nat \fun \family~\Rinf
\where
	\forall n: \nat @ \\
	\t1	\ballsRn(n) =  \{~ v: \Rtuples(n); r: \R @ \ballRn(n,v,r) ~\}
\end{axdef}

\subsection{\zcmd{tauRn}}

The usual topology on $\Rtuples(n)$ is the topology generated by the open balls in $\Rtuples(n)$.
Let $\tauRn(n)$ denote the usual topology on $\Rtuples(n)$.

\begin{axdef}
	\tauRn: \nat \fun \family~\Rinf
\where
	\forall n: \nat @ \\
	\t1	\tauRn(n) = topGen[\Rtuples(n)] (\ballsRn(n))
\end{axdef}

\begin{remark}

If $n \in \nat$ then $\tauRn(n)$ is a topology on $\Rtuples(n)$.

\begin{zed}
	\forall n: \nat @ \tauRn(n) \in top[\Rtuples(n)]
\end{zed}
\end{remark}

\subsection{\zcmd{Rtaun}}

Let $\Rtaun(n)$ denote the topological space defined by the usual topology on $\Rtuples(n)$.

\begin{axdef}
	\Rtaun: \nat \fun topSpaces[\Rinf]
\where
	\forall n: \nat @ \\
	\t1	\Rtaun(n) = (\Rtuples(n), \tauRn(n))
\end{axdef}

\section{Continuity}

A mapping $f$ from $\Rtuples(n)$ to $\Rtuples(m)$ is said to be continuous if it is continuous with respect to the usual topologies.
Let $\CzeroRnm(n,m)$ denote the set of these continuous mappings.

\begin{axdef}
	\CzeroRnm: \nat \cross \nat \fun \power(\Rinf \pfun \Rinf)
\where
	\forall n, m: \nat @ \\
	\t1	\CzeroRnm(n,m) = \CzeroTT(\Rtaun(n), \Rtaun(m)) 
\end{axdef}

\section{Differentiability}

Let $f: \R^n \fun \R^m$ and let $x \in \R^n$.
Then $f$ is differentiable at $x$ if there exists a linear transformation $L: \R^n \fun \R^m$
such that $f$ is approximately linear very near $x$.
$$
f(x + h) \approx f(x) + L(h) \quad \text{when} \quad \norm{h} \approx 0
$$ 

Let $\smoothRnm(x,n,m)$ denote the set of all functions $f \in \Rtuples(n) \pfun \Rtuples(m)$ that are smooth at $x \in \Rtuples(n)$.

\printbibliography

\end{document}