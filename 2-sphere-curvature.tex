\documentclass[11pt, oneside]{article}

\usepackage{shared/preamble}
\addbibresource{shared/references.bib}

\usepackage{real-numbers/real-numbers}
\usepackage{vector-spaces/vector-spaces}
\usepackage{2-sphere-curvature}

\title{Proof that $\Stwo$ is curved}
\author{Arthur Ryman, {\tt arthur.ryman@gmail.com}}
\date{\today}

\begin{document}

\maketitle

\begin{abstract}
Leonard Susskind,  in Lecture 3 of his Stanford University course on General Relativity,
asks his class to prove, as an exercise, that $\Stwo$ is curved.
This article attempts to do so.
The approach taken here is to precisely formalize every concept involved in the proof.
\end{abstract}

\section{Introduction}

The 2-sphere, $\Stwo$, is an extremely ubiquitous and illuminating example of a Riemannian manifold.
It appears in many interesting contexts including complex analysis, special relativity, electromagnetism, and quantum mechanics.
It is a compact submanifold of $\Rthree$ and so, unlike its higher-dimensional versions, can be clearly visualized.
These attributes make $\Stwo$ an ideal concrete example for illustrating the concepts of 
abstract Riemannian geometry.

In Lecture 3 of his Stanford University course on General Relativity \cite{susskind-gr3},
Leonard Susskind asks his class to prove that $\Stwo$ is curved.
He does this before defining curvature.
The class is simply tasked to show that there is no coordinate system in which the components
of the metric tensor become those of flat space.
The goal of this article is to produce such a proof.

\section{Proof Approaches}

Susskind takes a very concrete, coordinate system based approach to defining the objects of study in General Relativity.
He defines tensors as collections of real-valued functions defined for each coordinate system with the property that they
transform in prescribed ways under a change of coordinate system.
The primary object is the metric tensor.
The task is to show that there is no coordinate system on the sphere that makes its metric tensor the same as that of the flat plane.

\subsection{The Coordinate System Approach}
 
The most direct route is to attempt to construct a coordinate system that makes the metric flat.
This approach will lead to differential equations that the coordinate functions must satisfy.
I suspect that one can show that the differential equations have no solution, probably due to the failure of some
integrability condition.

For example, suppose one is given two smooth functions $f(x,y)$ and $g(x,y)$, and one is asked to find $F(x,y)$ such that
\begin{align}
	\frac{\partial F}{\partial x}	&= f \\
	\frac{\partial F}{\partial y}	&= g
\end{align}
If $F$ exists then $f$ and $g$ must satisfy the following integrability condition
\begin{equation}
\begin{split}
	\frac{\partial f}{\partial y} 	& = \frac{\partial}{\partial y} \left( \frac{\partial F}{\partial x} \right) \\
						& = \frac{\partial^2 F}{\partial y \partial x} \\
						& = \frac{\partial^2 F}{\partial x \partial y} \\
						& = \frac{\partial}{\partial x} \left( \frac{\partial F}{\partial y} \right) \\
						& = \frac{\partial g}{\partial x}
	\end{split}
\end{equation}

\subsection{The Curvature Tensor Approach}

The idea here is to define a tensor that vanishes on flat space but not on $\Stwo$.
The existence of such a tensor proves that $\Stwo$ is not flat.
Of course, the tensor in question is the Riemann curvature tensor but Susskind has not introduced it yet.

Geometrically, the curvature tensor arises when one considers the parallel transport of a tangent vector around a
small closed loop anchored at some point.
How hard would it be to define this tensor from first principles?
Let $M$ be a smooth manifold, let $p \in M$ be some point, and let $M_p$ denote the tangent space at $p$.
Let $V \in M_p$ be a tangent vector.
Let $X, Y$ be tangent vectors at $p$.
Consider the small path anchored at $p$ defined by moving along $X$, then $Y$, then $-X$, and then finally $-Y$.
Transport $V$ along this closed path.
The result is a new vector $W$ which is a linear transform of $V$ that depends linearly on $X$ and $Y$.
\begin{equation}
	W = R_p(X,Y)V
\end{equation}

One suspects that the amount of change of $V$ is proportional to the two-dimensional area spanned by $X$ and $Y$, hence $R$ depends
on the area element $X \wedge Y$, or 
\begin{equation}
	R_p(X,Y) = -R_p(Y,X)
\end{equation}

The tricky bit here is to define parallel transport.
It probably boils down to geodesic movement.

Parallel transport has an intuitive, physical meaning.
Suppose we have a massive, rigid body at rest at the point $p$.
By this I mean that the centre of mass of the body coincides with $p$.
Such a body defines a frame of reference at $p$.
We take $p$ to be the origin and pick several reference points on the body such that their displacements from $p$ define a set of basis vectors.
We give the body a slight push in some direction $V$, being careful not induce any rotation.
We achieve this by apply the same force in the direction $V$ at several points on the body, balancing them about the body's centre of mass.
We then let the body move a small distance in the direction defined by $V$ to some new point $p'$.
The new positions of the reference points define a new set of basis vectors.
For small movements the new basis vector will be defined by a linear transformation of the old basis vectors.

We can determine the equations of motion by defining a Hamiltonian.
It is natural to use the metric tensor as defining the kinetic energy of the system.
Solving the Hamiltonian equations of motion should yield a geodesic through $p$ in the direction of $V$ and the linear transformation of
the frame of reference.

Dirac, in his book on General Relativity, defined parallel transport for submanifolds of  $\R^n$ as the projections onto the submanifold of
straight line motion in $\R^n$.

\subsection{A Topological Approach}

Before proceeding with the coordinate approach, it is illuminating to discuss a purely topological one.
If $\Stwo$ is flat then its tangent bundle is trivial and so has a non-zero cross section.
But a cross section of the tangent bundle is a vector field on $\Stwo$.
A vector field on $\Stwo$ defines a diffeomorphism of $\Stwo$ that is homotopic to the identity map.
If the vector field never vanishes then this diffeomorphism has no fixed points.
The Lefschetz fixed-point theorem tells us that any space $M$ that has such a diffeomorphism must have a vanishing Euler characteristic
$\chi(M) = 0$. But $\chi(\Stwo) = 2$. Therefore $\Stwo$ cannot be flat.

While this proof answers the question, it doesn't provide much insight into curvature in general.
For example, the 3-sphere $\Sthree$ must surely also be curved, but $\chi(\Sthree) = 0$.
Curvature is a local, geometric property, not a global, topological one.
A coordinate proof for $\Stwo$ should generalize to all higher-dimensional spheres and to other submanifolds of $\R^n$.

\section{Riemannian Manifolds}

This section should be moved to an article dedicated to manifolds.

\subsubsection{\zcmd{family}}

Let $X$ be a set.
A {\it family} of subsets of $X$ is a set of subsets of $X$.
Let $\family X$ denote the set of all families of subsets of $X$.

\begin{zed}
	\family X == \power(\power X)
\end{zed}

\subsubsection{$Topology$}

A {\it topology} $\tau$ on $X$ is a family of subsets of $X$, referred to as the {\it open} subsets of $X$, that satisfy the following axioms.

\begin{schema}{Topology}[X]
	\tau: \family X
\where
	\emptyset \in \tau
\also
	X \in \tau
\also
	\forall F: \finset \tau @ \bigcap F \in \tau
\also
	\forall F: \power \tau @ \bigcup F \in \tau
\end{schema}

\begin{itemize}
\item The empty set is open.
\item The whole set is open.
\item The intersection of a finite family of open sets is open.
\item The union of any family of open sets is open. 
\end{itemize}

\subsubsection{$top$}

Let $top~X$ denote the set of all topologies on $X$.

\begin{zed}
	top~X == \{~ Topology[X] @ \tau ~\}
\end{zed}

\subsubsection{$discrete$ and $indiscrete$}

The {\it discrete} topology on $X$ consists of all subsets of $X$.
The {\it indiscrete} topology on $X$ consists of just $X$ and $\emptyset$.
Let $discrete(X)$ and $indiscrete(X)$ denote the discrete and indiscrete topologies on $X$.

\begin{zed}
	discrete~X == \power X
\also
	indiscrete~X ==  \{ \emptyset, X \}
\end{zed}

For example

\begin{zed}
	discrete~\nat \in top~\nat 
\also
	indiscrete~\nat \in top~\nat
\end{zed}

\subsubsection{$topgen$}

The intersection of a set of topologies on $X$ is also a topology on $X$.
Given a family $B$ of subsets of $X$, the topology {\it generated by} $B$ is the intersection of all
topologies that contain $B$.
The set $B$ is referred to as a {\it basis} for the topology it generates.
Let $topgen~X~B$ denote the topology on $X$ generated by the basis $B$.

\begin{zed}
	topgen~X == (\lambda B: \family X @ \bigcap \{~ \tau: top(X) | B \subseteq \tau ~\})
\end{zed}

For example

\begin{zed}
	(topgen~\nat) \{ \emptyset \} = indiscrete~\nat
\end{zed}

\section{Plan}

\begin{enumerate}

\item Define the usual topology on $\R^n$.

\item Define the smooth functions on $\R^n$.

\item Define tangent vectors.

\item Define the metric tensor and how its components transform under coordinate change.

\item Set up the differential equations to solve.

\item Solve the equations modulo terms of $O(x)$, $O(x^2)$, and $O(x^3)$, showing that no solution exists
modulo $O(x^3)$.

\item Generalize to other submanifolds of $R^3$.

\item Generalize to submanifolds of $R^n$ for $n > 3$.

\end{enumerate}


An $n$-dimensional manifold is a topological space that is locally homeomorphic to $\R^n$.

\subsection{\zcmd{Rtwo} and \zcmd{Rthree}}

Let $\Rtwo$ and $\Rthree$ denote the vector spaces of 2-tuples and 3-tuples.
\begin{zed}
	\Rtwo == \Rtuples(2) \\
	\Rthree == \Rtuples(3)
\end{zed}


\section{The 2-sphere as a Riemannian Manifold}

Let $\Stwo$ denote the 2-sphere of unit vectors in $\Rthree$.
\begin{equation}
	\Stwo = \{~ v: \Rthree | \norm{v} = 1 ~\}
\end{equation}

\begin{axdef}
	\Stwo: \power \Rthree
\where
	\Stwo = \{~ v: \Rthree | \normR(v) = \oneR ~\}
\end{axdef}

\subsection{Coordinates on $\Rthree$}

Let $\langle e_1, e_2, e_3 \rangle$ denote the standard orthonormal basis of $\Rthree$.
\begin{align}
e_1 & = \langle1, 0, 0 \rangle \\
e_2 & = \langle 0, 1, 0 \rangle \\
e_3 & = \langle 0, 0, 1 \rangle
\end{align}
\begin{axdef}
	e_1, e_2, e_3: \Rthree
\where
	e_1 = \langle~ \oneR, \zeroR, \zeroR ~\rangle \\
	e_2 = \langle~ \zeroR, \oneR, \zeroR ~\rangle \\
	e_3 = \langle~ \zeroR, \zeroR, \oneR ~\rangle
\end{axdef}

Let $x, y, z$ denote the standard Cartesian coordinate functions on $\Rthree$.
\begin{equation}
\forall p : \Rthree @ p = \langle x(p), y(p), z(p) \rangle
\end{equation}
\begin{axdef}
	x, y, z: \Rthree \fun \R
\where
	x = (\lambda p: \Rthree @ e_1 \dotR p) \\
	y = (\lambda p: \Rthree @ e_2 \dotR p) \\
	z = (\lambda p: \Rthree @ e_3 \dotR p)
\end{axdef}
\begin{remark}
We have $p = x(p) e_1 + y(p) e_2 + z(p) e_3$.
\end{remark}

The Cartesian coordinate functions define the vector fields $\frac{\partial}{\partial x}, \frac{\partial}{\partial y}, \frac{\partial}{\partial z}$
on $\Rthree$ as follows. 
For $p \in \Rthree$ and $f \in \smoothPR(\Rthree)$, we define
\begin{align}
\left(\frac{\partial}{\partial x}\right)_p f = \left. \frac{d}{dt} f(p + t e_1) \right|_{t = 0} \\
\left(\frac{\partial}{\partial y}\right)_p f = \left. \frac{d}{dt} f(p + t e_2) \right|_{t = 0} \\
\left(\frac{\partial}{\partial z}\right)_p f = \left. \frac{d}{dt} f(p + t e_3) \right|_{t = 0}
\end{align}

Let $\gamma$ be a smooth curve in $\Rthree$ and let $\gamma(0) = p$.
Let $\gamma'$ be the tangent vector at $p$ defined by $\gamma$.

Let $\gamma(p,v)$ denote the curve through the point $p \in \Rthree$ in the direction $v \in \Rthree$.
\begin{axdef}
	\gamma: \Rthree \cross \Rthree \fun (\R \fun \Rthree)
\where
	\forall p, v: \Rthree; t: \R @ \\
	\t1	\gamma(p,v)(t) = p \vaddR t \smulR v
\end{axdef}

Let $X, Y, Z$ denote the vector fields

\printbibliography

\end{document}