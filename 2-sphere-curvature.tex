\documentclass[11pt, oneside]{article}

\usepackage{geometry}
\geometry{letterpaper}

\usepackage{amsmath}
\usepackage{amsfonts}
\usepackage{amssymb}

\usepackage{commath}

\usepackage{fuzz}

% real numbers
\newcommand{\R}{{\mathbb{R}}}

\newcommand{\Rzero}{0}
\newcommand{\Rone}{1}

\newcommand{\Rneg}{-}

\newcommand{\Radd}{\mathbin{+}}
\newcommand{\Rsub}{\mathbin{-}}
%%inop \Radd \Rsub 3

\newcommand{\Rmul}{\mathbin{*}}
\newcommand{\Rdiv}{\mathbin{/}}
%%inop \Rmul \Rdiv 4

\newcommand{\Rlt}{\mathrel{<}}
\newcommand{\Rle}{\mathrel{\leq}}
\newcommand{\Rgt}{\mathrel{>}}
\newcommand{\Rge}{\mathrel{\geq}}
%%inrel \Rlt \Rle \Rgt \Rge

\DeclareMathOperator{\absR}{abs}
\DeclareMathOperator{\sqrtR}{sqrt}
\DeclareMathOperator{\openR}{open}
\DeclareMathOperator{\neighR}{neigh}
\DeclareMathOperator{\limR}{lim}

% real vector spaces
\newcommand{\Rinf}{\R^\infty}
\newcommand{\Rtuples}{\R}

\newcommand{\scalarMulRinf}{\mathbin{*}}
%%inop \scalarMulRinf 4

\newcommand{\vectorAddRinf}{\mathbin{+}}
\newcommand{\vectorSubRinf}{\mathbin{-}}
%%inop \vectorAddRinf \vectorSubRinf 3

\newcommand{\dotRinf}{\mathbin{\cdot}}
%%inop \dotRinf 4

\DeclareMathOperator{\normRinf}{norm}

\newcommand{\Rthree}{\R^3}

\title{Proof that $S^2$ is curved}
\author{Arthur Ryman, {\tt arthur.ryman@gmail.com}}
\date{\today}

\begin{document}

\maketitle

\begin{abstract}
Leonard Susskind asks his class to prove that $S^2$ is curved as an exercise in Lecture 3 of his
Stanford University course on General Relativity.
This article attempts to do so.
The approach taken here is to precisely formalize every concept involved in the proof.
\end{abstract}

\section{Introduction}

The 2-sphere, $S^2$, is an extremely important example of a Riemannian manifold.
It appears in many interesting contexts. 
It is a compact submanifold of $\R^3$ and can therefore be clearly visualized.
These attributes make $S^2$ an ideal concrete example for illustrating the concepts of 
abstract Riemannian geometry.

In Lecture 3 of his Stanford University course on General Relativity,
Susskind challenges his class to prove that $S^2$ is curved.
He does this before defining curvature.
The class is simply challenged to show that there is no coordinate system in which the components
of the metric tensor become those of flat space.

The goal of this article is to prove that $S^2$ is curved.
In doing so, I will precisely formalize all definitions using Z Notation and will validate
this document using the \fuzz\ type checker.

\section{Real Analysis}

The integers are built-in to Z Notation.
In principle, I could build up rational numbers from the integers,
and real numbers from the rational numbers, but that would take a lot of time and not really
accomplish much.
I'll assume that the real numbers are sufficiently simple and need no formalization.
I therefore define real numbers as a given set and then declare the usual constants and operations of real arithmetic.

Note that although the integers and reals are distinct types in the sense of typed set theory and Z Notation, mathematicians regard the integers as a subset of the reals.
I will therefore use the same symbols for real numbers and operations as for their integer counterparts even though
they are defined as distinct Z Notation variables.
For example, the integer $0$ looks the same typographically as the real number $\Rzero$ 
but the appropriate Z Notation variable will be used to make all formal text type check correctly.

\subsection{The Real Numbers}

Let $\R$ be the given set of real numbers.
\begin{zed}
	[\R]
\end{zed}

Let $\Rzero$ and $\Rone$ be the usual constants.
\begin{axdef}
	\Rzero: \R \\
	\Rone: \R
\end{axdef}

The usual negative operator has the following signature.
\begin{axdef}
	\Rneg: \R \fun \R
\end{axdef}

The usual arithmetic operators have the following signatures.
\begin{axdef}
	\_ \Radd \_: \R \cross \R \fun \R \\
	\_ \Rsub \_: \R \cross \R \fun \R \\
	\_ \Rmul \_: \R \cross \R \fun \R \\
	\_ \Rdiv \_: \R \cross \R \pfun \R
\where
	\dom(\_ \Rdiv \_) = \{~x, y: \R | y \neq \Rzero~\}
\end{axdef}

The usual comparison relations have the following signatures.
\begin{axdef}
	\_ \Rlt \_: \R \rel \R \\
	\_ \Rle \_: \R \rel \R \\
	\_ \Rgt \_: \R \rel \R \\
	\_ \Rge \_: \R \rel \R
\end{axdef}

Let $\absR{x}$ be the absolute value of the real number $x$.
\begin{axdef}
	\absR: \R \fun \R
\where
	\forall x: \R @ \absR(x) = \IF x \Rge \Rzero \THEN x \ELSE \Rneg~x
\end{axdef}

Let $\sqrtR{x}$ be the non-negative square root of the non-negative real number $x$.
\begin{axdef}
	\sqrtR: \R \pfun \R
\where
	\sqrtR = \{~ y: \R | y \Rge \Rzero @ y \Rmul y \mapsto y ~\}
\end{axdef}

\subsection{Open Intervals}

Let $a$ and $b$ be real numbers.
The open interval $\openR(a,b)$ defined by $a$ and $b$ is the set of all real numbers between them.
Clearly, this set is empty unless $a \Rlt b$.

\begin{axdef}
	\openR: \R \cross \R \fun \power \R
\where
	\forall a, b: \R @ \\
	\t1	\openR(a,b) = \{~ x: \R | a \Rlt x \Rlt b ~\}
\end{axdef}

\subsection{Neighbourhoods}

Any open interval that contains a real number $x$ is called a neighbourhood of it.
Clearly, every real number has an infinity of neighbourhoods.

\begin{axdef}
	\neighR: \R \fun \power (\power \R)
\where
	\forall x: \R @ \\
	\t1	\neighR(x) = \{~ a, b: \R | a \Rlt x \Rlt b @ \openR(a,b) ~\}
\end{axdef}

\subsection{Limits}

Let $U$ be a neighbourhood of the real number $c$.
Suppose that $f$ is a real-valued function defined everywhere on $U$ except possibly at $c$.
Then $f$ is said to approach the limit $L$ at $c$, $\lim_{x \to c}{f(x)} = L$,
if $f(x)$ gets
arbitrarily close to $L$ (within $\epsilon$ of $L$ for any specified $\epsilon > 0$)
for points $x$ sufficiently close to $c$ (within $\delta$ of $c$ for some $\delta > 0$) .

\begin{schema}{Limit}
	f: \R \pfun \R \\
	c, L: \R \\
	U: \power_1 \R
\where
	U \in \neighR(c)
\also
	U \setminus \{c\} \subseteq \dom f
\also
	\forall \epsilon: \R | \epsilon \Rgt \Rzero @ \\
	\t1	\exists \delta: \R | \delta \Rgt \Rzero @ \\
	\t2		\forall x: \dom f | \Rzero \Rlt \absR(x \Rsub c) \Rlt \delta @ \\
	\t3			\absR(f(x) \Rsub L) \Rlt \epsilon
\end{schema}
Clearly, if $f$ approaches a limit at $c$ then the limit is unique.
Let $\limR(c,L)$ be the set of all functions $f$ that approach the limit $L$ at $c$.

\begin{axdef}
	\limR: \R \cross \R \fun \power(\R \pfun \R)
\where
	\forall c, L: \R @ \\
	\t1	\limR(c,L) = \{f: \R \pfun \R; U: \neighR(c) | Limit @ f ~\}
\end{axdef}

\subsection{Continuity}

\subsection{Differentiability}

Let $f$ be a real-valued function on the real numbers and let $x$ be a real number.
The function $f$ is differentiable at $x$ if there exists a real number $f'(x)$ such that the following limit holds.
$$
\lim_{h \to 0} \frac{f(x+h) - f(x)}{h} = f'(x)
$$

The intuition behind this definition is that $f$ is approximately a linear function very near $x$.
$$
f(x + h) \approx f(x) + f'(x) h \quad \text{when} \quad \abs{h} \approx 0
$$

\section{Real Vector Spaces}

Since $S^2$ is a submanifold of $\R^3$, I need to start by defining $\R^3$ and the usual concepts 
associated with real vector spaces.

\subsection{Real $n$-tuples}

A finite sequence of $n \ge 0$ real numbers is called a real $n$-tuple.
Let $\Rinf$ be the set of all real $n$-tuples for any $n$.
\begin{zed}
	\Rinf == \seq \R
\end{zed}

Let $\Rtuples(n)$ be the set of all $n$-tuples for a given $n$.

\begin{axdef}
	\Rtuples: \nat \fun \power \Rinf
\where
	\forall n: \nat @ \\
	\t1	\Rtuples(n) = \{~ v: \Rinf | \# v = n ~\}
\end{axdef}

The real numbers that comprise an $n$-tuple are called its components.
The real number $v(i)$ is the $i$-th component of the $n$-tuple $v$ where
$1 \le i \le n$.
Let $\pi(i)$ be the projection function that maps an $n$-tuple $v$ to its $i$-th component $v(i)$.

\begin{axdef}
	\pi: \nat_1 \fun \Rinf \pfun \R
\where
	\forall i: \nat_1 @ \\
	\t1	\pi(i) = (\lambda v: \Rinf | i \in \dom v @ v(i))
\end{axdef}

\subsection{Scalar Multiplication}

Let $v$ be an $n$-tuple and let $c$ be a real number.
Scalar multiplication of $v$ by $c$ is the $n$-tuple $c \scalarMulRinf v$ defined by component-wise multiplication.

\begin{axdef}
	\_ \scalarMulRinf \_ : \R \cross \Rinf \fun \Rinf 
\where
	\forall c: \R @ \\
	\t1	c \scalarMulRinf \langle \rangle = \langle \rangle
\also
	\forall n: \nat_1 @ \\
	\t1	\forall c: \R; v: \Rtuples(n); i: 1 \upto n @ \\
	\t2		(c \scalarMulRinf v)(i) = c \Rmul v(i)
\end{axdef}

\subsection{Vector Addition and Subtraction}

Let $v$ and $w$ be $n$-tuples.
Vector addition of $v$ and $w$ is the $n$-tuple $v \vectorAddRinf w$ defined by component-wise addition.

\begin{axdef}
	\_ \vectorAddRinf \_: \Rinf \cross \Rinf \pfun \Rinf
\where
	\dom(\_ \vectorAddRinf \_) = \{~ v, w: \Rinf | \# v = \# w ~\}
\also
	\langle \rangle \vectorAddRinf \langle \rangle = \langle \rangle
\also
	\forall n: \nat_1 @ \\
	\t1	\forall v, w: \Rtuples(n); i: 1 \upto n @ \\
	\t2		(v \vectorAddRinf w)(i) = v(i) \Radd w(i)
\end{axdef}

Vector subtraction is defined similarly.

\begin{axdef}
	\_ \vectorSubRinf \_: \Rinf \cross \Rinf \pfun \Rinf
\where
	\dom(\_ \vectorSubRinf \_) = \{~ v, w: \Rinf | \# v = \# w ~\}
\also
	\langle \rangle \vectorSubRinf \langle \rangle = \langle \rangle
\also
	\forall n: \nat_1 @ \\
	\t1	\forall v, w: \Rtuples(n); i: 1 \upto n @ \\
	\t2		(v \vectorSubRinf w)(i) = v(i) \Rsub w(i)
\end{axdef}

Each $\Rtuples(n)$ is a real vector space under the operations of scalar multiplication and vector addition
defined above. 

\subsection{The Dot Product}

The inner or dot product of $n$-tuples $v$ and $w$ is the real number $v \dotRinf w$ defined by the sum of the component-wise products.

\begin{axdef}
	\_ \dotRinf \_ : \Rinf \cross \Rinf \pfun \R
\where
	\dom(\_ \dotRinf \_) = \{~ v, w: \Rinf | \# v = \# w ~\}
\also
	\langle \rangle \dotRinf \langle \rangle = \Rzero
\also
	\forall x, y: \R; v, w: \Rinf | \# v = \# w @ \\
	\t1	(\langle x \rangle \cat v) \dotRinf (\langle y \rangle \cat w) = x \Rmul y \Radd v \dotRinf w
\end{axdef}

Each $\Rtuples(n)$ is a real inner product space under the operation of dot product defined above.

\subsection{The Norm}

The norm $\norm{v}$ of an $n$-tuple $v$ is the positive square root of its dot product with itself.
$$
	\norm{v} = \sqrt{v \dotRinf v}
$$

\begin{axdef}
	\normRinf: \Rinf \fun \R
\where
	\forall v: \Rinf @ \\
	\t1	\normRinf(v) = \sqrtR(v \dotRinf v)
\end{axdef}

\subsection{Differentiability}

The concept of differentiability extends to functions between vector spaces that have norms.
Let $f: \R^n \fun \R^m$ and let $x \in \R^n$.
Then $f$ is differentiable at $x$ if there exists a linear transformation $(Df)(x): \R^n \fun \R^m$
such that $f$ is approximately linear very near $x$.
$$
f(x + h) \approx f(x) + ((Df)(x))(h) \quad \text{when} \quad \norm{h} \approx 0
$$ 

\section{The 2-sphere as a Riemannian Manifold}

Consider the vector space of 3-tuples, $\Rthree$.
\begin{zed}
	\Rthree == \Rtuples(3)
\end{zed}

\end{document}