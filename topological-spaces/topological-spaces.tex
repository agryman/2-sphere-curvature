\documentclass[11pt, oneside]{article}

\usepackage{../shared/preamble}
\addbibresource{../shared/references.bib}

\usepackage{topological-spaces}

\title{Topological Spaces}
\author{Arthur Ryman, {\tt arthur.ryman@gmail.com}}
\date{\today}

\begin{document}

\maketitle

\begin{abstract}
This article defines topological spaces and related concepts.
\end{abstract}

\section{Topological Spaces}

\subsection{$t_1$, $t_2$, and $t_3$}

Let $t_1$, $t_2$, and $t_3$ denote arbitrary sets.
These will be used throughout in the statement of theorems, remarks, and examples that are parameterized
by arbitrary sets.

\begin{zed}
	[t_1, t_2, t_3]
\end{zed}

\subsection{\zcmd{family}}

Let $X$ be a set.
A {\it family} of subsets of $X$ is a set of subsets of $X$.
Let $\family X$ denote the set of all families of subsets of $X$.

\begin{zed}
	\family X == \power(\power X)
\end{zed}

\subsection{$Topology$}

A {\it topology} $\tau$ on $X$ is a family of subsets of $X$, referred to as the {\it open} subsets of $X$, that satisfy the following axioms.

\begin{schema}{Topology}[X]
	\tau: \family X
\where
	\emptyset \in \tau
\also
	X \in \tau
\also
	\forall F: \finset \tau @ \bigcap F \in \tau
\also
	\forall F: \power \tau @ \bigcup F \in \tau
\end{schema}

\begin{itemize}
\item The empty set is open.
\item The whole set is open.
\item The intersection of a finite family of open sets is open.
\item The union of any family of open sets is open. 
\end{itemize}

\subsection{$top$ and $tops$}

Let $top[X]$ denote the set of all topologies on $X$.

\begin{gendef}[X]
	top: \power(\family X)
\where
	top = \{~ Topology[X] @ \tau ~\}
\end{gendef}

Let $tops[X]$ denote the set of all topologies on subsets $U \subseteq X$.

\begin{gendef}[X]
	tops: \power(\family X)
\where
	tops = \bigcup \{~ U: \power X @ top[U] ~\}
\end{gendef}

\subsection{$discrete$ and $indiscrete$}

The {\it discrete} topology on $X$ consists of all subsets of $X$.
The {\it indiscrete} topology on $X$ consists of just $X$ and $\emptyset$.
Let $discrete[X]$ and $indiscrete[X]$ denote the discrete and indiscrete topologies on $X$.

\begin{gendef}[X]
	discrete, indiscrete: \family X
\where
	discrete = \power X
\also
	indiscrete =  \{ \emptyset, X \}
\end{gendef}

\begin{example}
Let $t_1$ be an arbitrary set.
Then $discrete[t_1]$ and $indiscrete[t_1]$ are topologies on $t_1$.

\begin{zed}
	discrete[t_1] \in top[t_1] 
\also
	indiscrete[t_1] \in top[t_1]
\end{zed}

\end{example}

\subsection{$topGen$}

\begin{remark}

The intersection of a set of topologies on $X$ is also a topology on $X$.

\end{remark}

Given a family $B$ of subsets of $X$, the topology {\it generated by} $B$ is the intersection of all
topologies that contain $B$.
The set $B$ is referred to as a {\it basis} for the topology it generates.
Let $topGen[X]~B$ denote the topology on $X$ generated by the basis $B$.

\begin{gendef}[X]
	topGen: \family X \fun top[X]
\where
	\forall B: \family X @ \\
	\t1	topGen~B = \bigcap \{~ \tau: top[X] | B \subseteq \tau ~\}
\end{gendef}

\begin{example}
Let $t_1$ be an arbitrary set.

\begin{zed}
	topGen[t_1] \emptyset = indiscrete[t_1]
\also
	topGen[t_1] \{ \emptyset \} = indiscrete[t_1]
\also
	topGen[t_1] \{ t_1 \} = indiscrete[t_1]
\end{zed}

\end{example}

\subsection{$topSpace$}

Let $X$ be a set.
A {\it topological space} is a pair $(X, \tau)$ where $\tau$ is a topology on $X$.
Let $topSpace[X]$ denote the set of all topological spaces $(X,\tau)$.

\begin{zed}
	topSpace[X] == \{~ \tau: top[X] @ (X, \tau) ~\}
\end{zed}

\begin{example}
Let $t_1$ be an arbitrary set.

\begin{zed}
	(t_1, indiscrete[t_1]) \in topSpace[t_1]
\also
	(t_1, discrete[t_1]) \in topSpace[t_1]
\end{zed}


\end{example}

\subsection{$topSpaces$}

Let $topSpaces[t]$ dentote the set of all topological spaces $(X,\tau)$ where $X$ is a subset of $t$.

\begin{gendef}[t]
	topSpaces: \power t \rel \family t
\where
	topSpaces = \{~ X: \power t; \tau: \family t | \tau \in top[X] ~\}
\end{gendef}

\begin{remark}

\begin{zed}
	topSpace[t_1] \subseteq topSpaces[t_1]
\end{zed}

\end{remark}

\section{Continuous Mappings}

Let $(X,\tau)$ and $(Y,\sigma)$ be topological spaces.

\subsection{$Continuous$}

A mapping $f \in X \fun Y$ is said to be {\it continuous} if the inverse image of every open set is open.

\begin{schema}{Continuous}[X,Y]
	f: X \fun Y \\
	\tau: top[X] \\
	\sigma: top[Y]
\where
	\forall U: \sigma @ \\
	\t1	f\inv\limg U \rimg \in \tau
\end{schema}

\subsection{\zcmd{CzeroTT}}

Let $A$ and $B$ be topological spaces, and
let $\CzeroTT(A,B)$ denote the set of continuous mappings from $A$ to $B$.

\begin{gendef}[X,Y]
	\CzeroTT: topSpace[X] \cross topSpace[Y] \fun \power (X \fun Y)
\where
	\forall \tau: top[X]; \sigma: top[Y] @ \\
	\t1	\LET A == (X, \tau); B == (Y, \sigma) @ \\
	\t2		\CzeroTT(A,B) = \{~ f: X \fun Y | Continuous[X,Y] ~\}
\end{gendef}

\subsection{The Identity Mapping}

\begin{remark}
The identity mapping is continuous.

\begin{zed}
	\forall \tau: top[t_1] @ \\
	\t1	\LET A == (t_1, \tau) @ \\
	\t2		\id t_1 \in \CzeroTT(A, A)
\end{zed}

\end{remark}

\subsection{\zcmd{const}}

Let $X$ and $Y$ be sets and let $c \in Y$ be some given point.
The mapping that sends every point of $X$ to $c$ is called the {\it constant mapping} defined by $c$.
Let $\const(c)$ denote the constant mapping.

\begin{gendef}[X,Y]
	\const: Y \fun (X \fun Y)
\where
	\forall c: Y @ \\
	\t1	\const(c) = (\lambda x: X @ c)
\end{gendef}

\begin{remark}
The constant mapping is continuous.

\begin{zed}
	\forall \tau: top[t_1]; \sigma: top[t_2]; c: t_2 @ \\
	\t1	\LET A == (t_1, \tau); B == (t_2, \sigma) @ \\
	\t2		\const[t_1,t_2] c \in \CzeroTT(A,B)
\end{zed}

\end{remark}

\subsection{Composition of Continuous Mapping}

\begin{remark}
Let $t_1$, $t_2$, and $t_3$ be arbitrary sets.
The composition of continuous mappings is a continuous mapping.

\begin{zed}
	\forall A: topSpace[t_1]; B: topSpace[t_2]; C: topSpace[t_3] @ \\
	\t1	\forall f: \CzeroTT(A, B); g: \CzeroTT(B, C) @ \\
	\t2		g \circ f \in \CzeroTT(A, C)
\end{zed}

\end{remark}

\section{Induced Topology}

Let $A = (X, \tau)$ be a topological space and let $U \subseteq X$ be a subset.
The topology on $X$ {\it induces} a topology on $U$.
This topology is variously referred to as the {\it induced}, {\it relative}, or {\it subspace} topology on $U$.

\subsection{\zcmd{inducedFam}}

Let $\phi$ be a family of subsets of $X$ and let $U$ be a subset of $X$.
The family of subsets of $U$ {\it induced} by $\phi$ is the set of intersections of the members of $\phi$ with $U$.
Let $\phi \inducedFam U$ denote the family on $U$ induced by $\phi$.

\begin{gendef}[X]
	\_ \inducedFam \_:  \family X \cross \power X \fun \family X
\where
	\forall \phi: \family X; U: \power X @ \\
	\t1	\phi \inducedFam U = \{~ Y: \phi @ Y \cap U ~\}
\end{gendef}

\begin{remark}
If $\tau$ is a topology on $X$ then $\tau \inducedFam U$ is a topology on $U$.

\begin{zed}
	\forall \tau: top[t_1]; U: \power t_1 @ \\
	\t1	\tau \inducedFam U \in top[U]
\end{zed}

\end{remark}

\subsection{\zcmd{inducedTop} and \zcmd{inducedTopSp}}

Let $A = (X, \tau)$ be a topological space.
Let $\tau \inducedTop U$ denote the topology on $U$ induced by $\tau$
\begin{gendef}[X]
	\_ \inducedTop \_: top[X] \cross \power X \fun tops[X]
\where
	\forall \tau: top[X]; U: \power X @ \\
	\t1	\tau \inducedTop U = \tau \inducedFam U
\end{gendef}

Let $A \inducedTopSp U$ denote the corresponding induced topological space.

\begin{gendef}[X]
	\_ \inducedTopSp \_: topSpace[X] \cross \power X \fun topSpaces[X]
\where
	\forall \tau: top[X]; U: \power X @ \\
	\t1	\LET A == (X, \tau) @ \\
	\t2		A \inducedTopSp U = (U, \tau \inducedTop U)
\end{gendef}

\begin{remark}
The induced topological space $A \inducedTopSp U$ is a topological space on $U$.

\begin{zed}
	\forall \tau: top[t_1]; U: \power t_1 @ \\
	\t1	\LET A == (t_1, \tau) @ \\
	\t2		A \inducedTopSp U \in topSpace[U]
\end{zed}

\end{remark}

\section{Product Topology}

Let $X$ and $Y$ be sets and let $A$ and $B$ be topological spaces on them.
There is a natural topology on the product set $X \cross Y$ generated by the products of the open sets on $X$ and $Y$.

\subsection{\zcmd{prodTop}}

Let $A \prodTop B$ denote the product topological space.

\begin{gendef}[X,Y]
	\_ \prodTop \_: topSpace[X] \cross topSpace[Y] \fun topSpace[X \cross Y]
\where
	\forall \tau: top[X]; \sigma: top[Y] @ \\
	\t1	\LET A == (X, \tau); B == (Y, \sigma); Z == X \cross Y @ \\
	\t2		A \prodTop B = (Z, topGen[Z](\tau \cross \sigma))
\end{gendef}

\printbibliography

\end{document}