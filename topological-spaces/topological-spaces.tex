\documentclass[11pt, oneside]{article}

\usepackage{../shared/preamble}
\addbibresource{../shared/references.bib}

\usepackage{topological-spaces}

\title{Topological Spaces}
\author{Arthur Ryman, {\tt arthur.ryman@gmail.com}}
\date{\today}

\begin{document}

\maketitle

\begin{abstract}
This article defines topological spaces and related concepts.
\end{abstract}

\section{Topological Spaces}

\subsection{$t_1$, $t_2$, and $t_3$}

Let $t_1$, $t_2$, and $t_3$ denote arbitrary sets.
These will be used throughout in the statement of theorems, remarks, and examples that are parameterized
by arbitrary sets.

\begin{zed}
	[t_1, t_2, t_3]
\end{zed}

\subsection{\zcmd{family}}

Let $X$ be a set.
A {\it family} of subsets of $X$ is a set of subsets of $X$.
Let $\family X$ denote the set of all families of subsets of $X$.

\begin{zed}
	\family X == \power(\power X)
\end{zed}

\subsection{$Topology$}

A {\it topology} $\tau$ on $X$ is a family of subsets of $X$, referred to as the {\it open} subsets of $X$, that satisfy the following axioms.

\begin{schema}{Topology}[X]
	\tau: \family X
\where
	\emptyset \in \tau
\also
	X \in \tau
\also
	\forall F: \finset \tau @ \bigcap F \in \tau
\also
	\forall F: \power \tau @ \bigcup F \in \tau
\end{schema}

\begin{itemize}
\item The empty set is open.
\item The whole set is open.
\item The intersection of a finite family of open sets is open.
\item The union of any family of open sets is open. 
\end{itemize}

\subsection{$top$}

Let $top[X]$ denote the set of all topologies on $X$.

\begin{zed}
	top[X] == \{~ Topology[X] @ \tau ~\}
\end{zed}

\subsection{$discrete$ and $indiscrete$}

The {\it discrete} topology on $X$ consists of all subsets of $X$.
The {\it indiscrete} topology on $X$ consists of just $X$ and $\emptyset$.
Let $discrete[X]$ and $indiscrete[X]$ denote the discrete and indiscrete topologies on $X$.

\begin{zed}
	discrete[X] == \power X
\also
	indiscrete[X] ==  \{ \emptyset, X \}
\end{zed}

\begin{example}
Let $t_1$ be an arbitrary set.

\begin{zed}
	discrete[t_1] \in top[t_1] 
\also
	indiscrete[t_1] \in top[t_1]
\end{zed}

\end{example}

\subsection{$topGen$}

\begin{remark}

The intersection of a set of topologies on $X$ is also a topology on $X$.

\end{remark}

Given a family $B$ of subsets of $X$, the topology {\it generated by} $B$ is the intersection of all
topologies that contain $B$.
The set $B$ is referred to as a {\it basis} for the topology it generates.
Let $topGen[X]~B$ denote the topology on $X$ generated by the basis $B$.

\begin{gendef}[X]
	topGen: \family X \fun top[X]
\where
	\forall B: \family X @ \\
	\t1	topGen~B = \bigcap \{~ \tau: top[X] | B \subseteq \tau ~\}
\end{gendef}

\begin{example}
Let $t_1$ be an arbitrary set.

\begin{zed}
	topGen[t_1] \emptyset = indiscrete[t_1]
\also
	topGen[t_1] \{ \emptyset \} = indiscrete[t_1]
\also
	topGen[t_1] \{ t_1 \} = indiscrete[t_1]
\end{zed}

\end{example}

\subsection{$topSpace$}

Let $X$ be a set.
A {\it topological space} is a pair $(X, \tau)$ where $\tau$ is a topology on $X$.
Let $topSpace[X]$ denote the set of all topological spaces $(X,\tau)$.

\begin{zed}
	topSpace[X] == \{~ \tau: top[X] @ (X, \tau) ~\}
\end{zed}

\begin{example}
Let $t_1$ be an arbitrary set.

\begin{zed}
	(t_1, indiscrete[t_1]) \in topSpace[t_1]
\also
	(t_1, discrete[t_1]) \in topSpace[t_1]
\end{zed}

\end{example}

\section{Continuous Mappings}

Let $(X,\tau)$ and $(Y,\sigma)$ be topological spaces.

\subsection{$Continuous$}

A mapping $f \in X \fun Y$ is said to be {\it continuous} if the inverse image of every open set is open.

\begin{schema}{Continuous}[X,Y]
	f: X \fun Y \\
	\tau: top[X] \\
	\sigma: top[Y]
\where
	\forall U: \sigma @ \\
	\t1	f\inv\limg U \rimg \in \tau
\end{schema}

\subsection{\zcmd{CzeroTT}}

Let $\CzeroTT((X,\tau),(Y,\sigma))$ denote the set of continuous mappings from $(X,\tau)$ to $(Y,\sigma)$.

\begin{gendef}[X,Y]
	\CzeroTT: topSpace[X] \cross topSpace[Y] \fun \power (X \fun Y)
\where
	\forall \tau: top[X]; \sigma: top[Y] @ \\
	\t1	\CzeroTT((X,\tau),(Y,\sigma)) = \{~ f: X \fun Y | Continuous[X,Y] ~\}
\end{gendef}

\begin{remark}
Let $t_1$, $t_2$, and $t_3$ be arbitrary sets.
The composition of continuous mappings is a continuous mapping.

\begin{zed}
	\forall A: topSpace[t_1]; B: topSpace[t_2]; C: topSpace[t_3] @ \\
	\t1	\forall f: \CzeroTT(A, B); g: \CzeroTT(B, C) @ \\
	\t2		g \circ f \in \CzeroTT(A, C)
\end{zed}

\end{remark}


\printbibliography

\end{document}