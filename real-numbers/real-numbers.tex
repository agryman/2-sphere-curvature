\documentclass[11pt, oneside]{article}

\usepackage[utf8]{inputenc}
\usepackage[english]{babel}

\usepackage{geometry}
\geometry{letterpaper}

\usepackage{amsmath}
\usepackage{amsfonts}
\usepackage{amssymb}
\usepackage{amsthm}

\usepackage{commath}

\usepackage[autostyle]{csquotes}

\usepackage{hyperref}

\usepackage[
backend=biber,
style=alphabetic,
sorting=ynt
]{biblatex}
\addbibresource{../shared/references.bib}

% put fuzz last because it redefines \t
\usepackage{fuzz}

% real numbers
\newcommand{\R}{\mathbb{R}}
\newcommand{\Rpos}{\R_{+}}
\newcommand{\Rnz}{\R_{*}}


\newcommand{\zeroR}{0}
\newcommand{\oneR}{1}

\newcommand{\negR}{-}

\newcommand{\addR}{\mathbin{+}}
\newcommand{\subR}{\mathbin{-}}
%%inop \addR \subR 3

\newcommand{\mulR}{\mathbin{*}}
\newcommand{\divR}{\mathbin{/}}
%%inop \mulR \divR 4

\newcommand{\ltR}{\mathrel{<}}
\newcommand{\leR}{\mathrel{\leq}}
\newcommand{\gtR}{\mathrel{>}}
\newcommand{\geR}{\mathrel{\geq}}
%%inrel \ltR \leR \gtR \geR

\DeclareMathOperator{\absR}{abs}
\DeclareMathOperator{\sqrtR}{sqrt}

\DeclareMathOperator{\intervalRR}{interval}
\DeclareMathOperator{\ballRR}{ball}
\DeclareMathOperator{\neighR}{neigh}

\DeclareMathOperator{\limRR}{lim}
\DeclareMathOperator{\limR}{lim}
\DeclareMathOperator{\limFR}{lim}

\DeclareMathOperator{\FunR}{F}
\DeclareMathOperator{\FunPR}{F}

\DeclareMathOperator{\CzeroR}{C^0}
\DeclareMathOperator{\CzeroPR}{C^0}

\DeclareMathOperator{\CnR}{C}
\DeclareMathOperator{\CnPR}{C}

\DeclareMathOperator{\diffRR}{diff}
\DeclareMathOperator{\diffR}{diff}
\DeclareMathOperator{\diffPR}{diff}

\DeclareMathOperator{\smoothR}{C^{\infty}}
\DeclareMathOperator{\smoothPR}{C^{\infty}}

\DeclareMathOperator{\derivFR}{deriv}
\DeclareMathOperator{\derivF}{D}

% real vector spaces
\newcommand{\Rinf}{\R^\infty}
\newcommand{\Rtuples}{\R}

\newcommand{\smulR}{\mathbin{*}}
%%inop \smulR 4

\newcommand{\vaddR}{\mathbin{+}}
\newcommand{\vsubR}{\mathbin{-}}
%%inop \vaddR \vsubR 3

\newcommand{\dotR}{\mathbin{\cdot}}
%%inop \dotR 4

\DeclareMathOperator{\normR}{norm}

\DeclareMathOperator{\linR}{lin}


\newcommand{\Rtwo}{\R^2}
\newcommand{\Rthree}{\R^3}

\newcommand{\Stwo}{S^2}
\newcommand{\Sthree}{S^3}
\newcommand{\Sn}{S^n}



\newtheorem{theorem}{Theorem}
\newtheorem*{remark}{Remark}

\title{Real Numbers}
\author{Arthur Ryman, {\tt arthur.ryman@gmail.com}}
\date{\today}

\begin{document}

\maketitle

\begin{abstract}
This article contains Z Notation type declarations for the real numbers, $\R$, and some related objects.
It has been type checked by \fuzz.
\end{abstract}

\section{Introduction}

The real numbers, $\R$, are foundational to many many mathematical objects such as vector spaces and manifolds,
but are not built-in to Z Notation.
The article provides type declarations for $\R$ and related objects so that they can be used and type checked in formal Z specifications.

No attempt has been made to provide axiomatic definitions of these objects since they would only be a use for proof checking.
Although proof checking is highly desirable, it is beyond the scope of this article.
The type declarations given here are intended to provide a basis for future axiomatization.

\section{The Real Numbers}

Let $\R$ denote the given set of real numbers.
\begin{zed}
	[\R]
\end{zed}

Let $\zeroR$ and $\oneR$ denote the zero and unit elements of the real numbers.
\begin{axdef}
	\zeroR: \R \\
	\oneR: \R
\end{axdef}

Define $\Rnz$ to be the set of non-zero real numbers, 
also referred to as the punctured real number line.
\begin{zed}
	\Rnz == \R \setminus \{ \zeroR \}
\end{zed}

The usual comparison relations have the following signatures.
\begin{axdef}
	\_ \ltR \_: \R \rel \R \\
	\_ \leR \_: \R \rel \R \\
	\_ \gtR \_: \R \rel \R \\
	\_ \geR \_: \R \rel \R
\end{axdef}

Define $\Rpos$ to be the set of positive real numbers.
\begin{zed}
	\Rpos == \{~ x: \R | x \gtR \zeroR ~\}
\end{zed}

The usual negative operator has the following signature.
\begin{axdef}
	\negR: \R \fun \R
\end{axdef}

Define $\absR{x}$ to be $\abs{x}$, the absolute value of the real number $x$.
\begin{axdef}
	\absR: \R \fun \R
\where
	\forall x: \R @ \absR(x) = \IF x \geR \zeroR \THEN x \ELSE \negR~x
\end{axdef}

The usual arithmetic operators have the following signatures.
\begin{axdef}
	\_ \addR \_: \R \cross \R \fun \R \\
	\_ \subR \_: \R \cross \R \fun \R \\
	\_ \mulR \_: \R \cross \R \fun \R \\
	\_ \divR \_: \R \cross \Rnz \fun \R
\end{axdef}

Define $\sqrtR{x}$ to be $\sqrt{x}$, the non-negative square root of the non-negative real number $x$.
\begin{axdef}
	\sqrtR: \R \pfun \R
\where
	\sqrtR = \{~ x: \R | x \geR \zeroR @ x \mulR x \mapsto x ~\}
\end{axdef}

\section{Open Intervals}

Let $a$ and $b$ be real numbers.
The open interval bounded by $a$ and $b$ is the set of all real numbers between $a$ and $b$.
Define $\intervalRR(a,b)$ to be $(a,b)$, the open interval bounded by $a$ and $b$. 

\begin{axdef}
	\intervalRR: \R \cross \R \fun \power \R
\where
	\forall a, b: \R @ \\
	\t1	\intervalRR(a,b) = \{~ x: \R | a \ltR x \ltR b ~\}
\end{axdef}
Clearly, $\intervalRR(a,b)$ is empty if $a \geR b$.

\section{Open Balls}
Let $x$ be a real number and let $r$ be a strictly positive real number.
Define $\ballRR(x,r)$ to the the open interval that contains all points within distance $r$ of $x$.
\begin{axdef}
	\ballRR: \R \cross \Rpos \fun \power \R
\where
	\forall x: \R; r: \Rpos @ \\
	\t1	\ballRR(x,r) = \{~ x': \R | \absR(x' \subR x) \ltR r ~\}
\end{axdef}
\begin{remark}
$\ballRR(x,r) = \intervalRR(x-r, x+r)$
\end{remark}

\section{Neighbourhoods}

Let $x$ be a real number.
Any open ball centred at $x$ is called a neighbourhood of it.
Define $\neighR(x)$ to be the set of all neighbourhoods of $x$.

\begin{axdef}
	\neighR: \R \fun \power (\power \R)
\where
	\forall x: \R @ \\
	\t1	\neighR(x) = \{~ r: \Rpos @ \ballRR(x,r) ~\}
\end{axdef}
Clearly, every real number has an infinity of neighbourhoods.

\section{Functions}

Our next goal is to define continuity, limits, and differentiability.
These are {\it local} properties of functions
in the sense that they only depend on the values that the function takes in an arbitrarily small neighbourhood of a given point.
We therefore restrict our attention to functions that are defined in some neighbourhood of each point in their domains.
Let $x$ be a real number. 
Define $\FunR(x)$ to be the set of all real-valued functions that are defined in some neighbourhood of $x$.
\begin{axdef}
	\FunR: \R \fun \power(\R \pfun \R)
\where
	\forall x: \R @ \\
	\t1	\FunR(x) = \{~ f: \R \pfun \R | \exists U: \neighR(x) @ U \subseteq \dom f ~\}
\end{axdef}

Let $U$ be a subset of $\R$.
Define $\FunPR(U)$ to be the set of a real-valued functions on $U$ that are defined in some neighbourhood of every point of $U$.
\begin{axdef}
	\FunPR: \power \R \fun \power (\R \pfun \R)
\where
	\forall U: \power \R @ \\
	\t1	\FunPR(U) = \{~ f: U \fun \R | \forall x: U @ f \in \FunR(x) ~\}
\end{axdef}

\section{Continuity}

Let $f$ be a real-valued function and 
let $x$ be a real number.
The function $f$ is said to be continuous at $x$ if 
the domain of $f$ contains some neighbourhood $U$ of $x$ such that
for any $\epsilon > 0$ there is some $\delta > 0$ for which 
$f(x')$ is always within $\epsilon$ of $f(x)$
when $x' \in U$ is within $\delta$ of $x$.
\begin{argue}
\forall \epsilon > 0 @ \exists \delta > 0 @ \forall x' \in U @ \\
\t1	\abs{x' - x} < \delta \implies \abs{f(x') - f(x)} < \epsilon
\end{argue}

\begin{schema}{Continuous}
	f: \R \pfun \R \\
	x: \R
\where
	f \in \FunR(x)
\also
	\forall \epsilon: \Rpos @ \exists \delta: \Rpos@ \forall x': \dom f @ \\
	\t1	\absR(x' \subR x) \ltR \delta \implies \absR(f(x') \subR f(x)) \ltR \epsilon
\end{schema}

Define $\CzeroR(x)$ to be the set of all functions that are continuous at $x$.
\begin{axdef}
	\CzeroR: \R \fun \power(\R \pfun \R)
\where
	\forall x: \R @ \\
	\t1	\CzeroR(x) = \{~ f: \R \pfun \R | Continuous ~\}
\end{axdef}

Let $U$ be any subset of $\R$. 
Define $\CzeroPR(U)$ to be the set of all functions on $U$ that are continuous at each point in $U$.
\begin{axdef}
	\CzeroPR: \power \R \fun \power (\R \pfun \R)
\where
	\forall U: \power \R @ \\
	\t1	\CzeroPR(U) = \{~ f: \FunPR(U) | \forall x: U @ f \in   \CzeroR(x) ~\}
\end{axdef}

\begin{remark}
If $f \in \CzeroPR(U)$ then $U$ is a, possibly infinite, union of neighbourhoods.
\end{remark}

\section{Limits}

Let $x$ and $l$ be real numbers and
let $f$ be a real-valued function that is defined everywhere in some
neighbourhood $U$ of $x$, except possibly at $x$.
The function $f$ is said to approach the limit $l$ at $x$ if $f \oplus \{ x \mapsto l \}$ is continuous at $x$.
$$
	\lim_{x' \to x}{f(x')} = l
$$

\begin{schema}{Limit}
	f: \R \pfun \R \\
	x, l: \R
\where
	f \oplus \{x \mapsto l\} \in \CzeroR(x)
\end{schema}

Let $\limRR(x,l)$ be the set of all functions that approach the limit $l$ at $x$.
\begin{axdef}
	\limRR: \R \cross \R \fun \power(\R \pfun \R)
\where
	\forall x, l: \R @ \\
	\t1	\limRR(x,l) = \{~ f: \R \pfun \R | Limit ~\}
\end{axdef}

\begin{theorem}
If a function $f$ approaches some limit at $x$ then that limit is unique.
\begin{zed}
	\forall x, l, l': \R @ \\
	\t1	\forall f : \limRR(x,l) \cap \limRR(x,l') @ \\
	\t2		l = l'
\end{zed}
\end{theorem}

\begin{proof}
Suppose we are given real numbers
\begin{argue}
	x, l, l' \in \R 
\end{argue}
and a function
\begin{argue}
	f \in \limRR(x,l) \cap \limRR(x,l')
\end{argue}
Let $\epsilon$ be any positive real number
\begin{argue}
	\epsilon > 0
\end{argue}
Since $f$ approaches limits $l$ and $l'$ at $x$ there exists a real number $\delta > 0$ such that
\begin{argue}
	\forall x' \in \R |  \\
	\t1	\zeroR \ltR \abs{x' \subR x}< \delta @ \\
	\t2		 \abs{f(x') - l} < \epsilon \land \abs{f(x') - l'} < \epsilon
\end{argue}
For any such real number $x'$ we have
\begin{argue}
	\abs{l' - l} \\
	\t1	= \abs{(f(x') - l) - (f(x') - l')} 		& add and subtract $f(x')$ \\
	\t1	\leq \abs{f(x') - l} + \abs{f(x') - l'} 	& triangle inequality \\
	\t1	= 2\epsilon					& definition of limits
\end{argue}
Since the above holds for any $\epsilon > 0$ we must have
\begin{argue}
	l = l'
\end{argue}

\end{proof}

If $f$ approaches the limit $l$ at $x$ then define $\limFR(f,x) = l$.
By the preceding theorem, $\limFR(f,x)$ is well-defined when it exists.
\begin{axdef}
	\limFR: (\R \pfun \R) \cross \R \pfun \R
\where
	\limFR = \{~ Limit @ (f, x) \mapsto l ~\}
\end{axdef}

\section{Differentiability}

Let $f$ be a real-valued function on the real numbers, let $x$ be a real number,
and let $f$ be defined on some neighbourhood $U$ of $x$.

The function $f$ is said to be differentiable at $x$ if the following limit holds for some number $f'(x)$.
$$
\lim_{h \to 0} \frac{f(x+h) - f(x)}{h} = f'(x)
$$

\begin{remark}
If $f$ is differentiable at $x$ then $f$ is continuous at $x$.
\end{remark}

The geometric intuition behind the concept of differentiability is that $f$ is differentiable at $x$
when, very near $x$, the function $f$ is approximately a straight line with slope $f'(x)$.
$$
f(x + h) \approx f(x) + f'(x) h \quad \text{when} \quad \abs{h} \approx 0
$$
The slope $f'(x)$ is called the derivative of $f$ at $x$.

We can read this definition as saying that the approximate slope function $m(h)$ defined for 
small enough, non-zero values of $h$ by
$$
	m(h) = \frac{f(x + h) - f(x)}{h}
$$
approaches the limit $l = f'(x)$ as $h \to 0$.
$$
	\lim_{h\to 0}{m(h)} = l = f'(x)
$$

\begin{schema}{Differentiable}
	f: \R \pfun \R \\
	x, l: \R
\where
	f \in \CzeroR(x)
\also
	\LET m == (\lambda h: \Rnz | x \addR h \in \dom f @ (f(x \addR h) \subR f(x)) \divR h) @ \\
	\t1	\limFR(m, \zeroR) = l
\end{schema}
\begin{remark}
If $f$ is differentiable at $x$ then $L$ is unique.
\end{remark}

Define $\diffRR(x,L)$ to be the set of all functions $f$ that are differentiable at $x$ with $l = f'(x)$.
\begin{axdef}
	\diffRR: \R \cross \R \fun \power(\R \pfun \R)
\where
	\forall x, l: \R @ \\
	\t1	\diffRR(x, l) = \{~ f: \R \pfun \R | Differentiable ~\}
\end{axdef}

Define $\diffR(x)$ to be the set of all functions that are differentiable at $x$.
\begin{axdef}
	\diffR: \R \fun \power(\R \pfun \R)
\where
	\forall x: \R @ \\
	\t1	\diffR(x) = \bigcup \{~ l: \R @ \diffRR(x,l) ~\}
\end{axdef}

Let $U$ be any subset of $\R$. 
Define $\diffPR(U)$ to be the set of all functions on $U$
that are differentiable at each point of $U$.
\begin{axdef}
	\diffPR: \power \R \fun \power(\R \pfun \R)
\where
	\forall U: \power \R @ \\
	\t1	\diffPR(U) = \{~ f: \CzeroPR(U) | \forall x: U @ f \in \diffR(x) ~\}
\end{axdef}

\section{Derivatives}

The function $f'$ is called the derived function or the derivative of $f$.
Define $\derivFR(f,x)$ to be $f'(x)$.
\begin{axdef}
	\derivFR: (\R \pfun \R) \cross \R \pfun \R
\where
	\derivFR = \{~ Differentiable @ (f,x) \mapsto l ~\}
\end{axdef}

Define $\derivF(f)$ to be the derived function $f'$.
\begin{axdef}
	\derivF: (\R \pfun \R) \fun (\R \pfun \R)
\where
	\forall f: \R \pfun \R @ \\
		\t1	\derivF f = (\lambda x: \R | f \in \diffR(x) @ \derivFR(f,x)) 
\end{axdef}

\begin{remark}
If $f$ is differentiable on $U$ then $f'$ is not necessarily continuous on $U$.
Counterexamples exist.
If $f$ is uniformly differentiable then $f'$ is continuous, but I won't discuss uniform differentiability further.
\end{remark}

\section{Higher Order Derivatives}

Let $n$ be a natural number and let $x$ be a real number.
In differential geometry we normally deal with $C^n(x)$, the set of functions
that possess continuous derivatives of order $0, \ldots, n$ at $x$.
Define $\CnR(n,x)$ to be the set of all functions that have continuous derivatives of order $0, \ldots, n$ at $x$.
\begin{axdef}
	\CnR: \nat \cross \R \fun \power(\R \pfun \R)
\where
	\forall x: \R @ \\
	\t1	\CnR(0,x) = \CzeroR(x)
\also
	\forall n: \nat; x: \R @ \\
	\t1	\CnR(n + 1, x) = \{~ f: \CzeroR(x) | \derivF f \in \CnR(n,x) ~\}
\end{axdef}

Let $n$ be a natural number and let $U$ be a subset of $\R$.
Define $\CnPR(n,U)$ to be the set of all functions on $U$ that have continuous derivatives of order $0, \ldots, n$
at every point of $U$.
\begin{axdef}
	\CnPR: \nat \cross \power \R \fun \power(\R \pfun \R)
\where
	\forall n: \nat; U: \power \R @ \\
	\t1	\CnPR(n,U) = \{~ f: \FunPR(U) | \forall x: U @ f \in \CnR(n,x) ~\}
\end{axdef}

\section{Smoothness}

A function is said to be smooth if it possesses continuous derivatives of all orders.
Let $x$ be a real number.
Define $\smoothR(x)$ to be the set of all functions that are smooth at $x$.
\begin{axdef}
	\smoothR: \R \fun \power(\R \pfun \R)
\where
	\forall x: \R @ \\
	\t1	\smoothR(x) = \{~ f: \FunR(x) | \forall n: \nat @ f \in \CnR(n, x) ~\}
\end{axdef}

Define $\smoothPR(U)$ to be the set of all functions on $U$ that are smooth at every point of $U$.
\begin{axdef}
	\smoothPR: \power \R \fun \power (\R \pfun \R)
\where
	\forall U: \power \R @ \\
	\t1	\smoothPR(U) = \{~ f: \FunPR(U) | \forall x: U @ f \in \smoothR(x) ~\}
\end{axdef}

\printbibliography

\end{document}